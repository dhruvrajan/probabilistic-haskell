\documentclass[12pt]{article}


\newcommand{\s}[1]{\texttt{#1}}
\renewcommand{\it}[1]{\textit{#1}}
\begin{document}

\title{Plan for Language Specification}
\author{Dhruv Rajan}
\date{\today}
\maketitle

\section{Figaro's Representation}
\subsection{Primitive Types}
In Figaro, every data structure is an \it{element}. Every element
holds a \textit{value}, with some associated \textit{value type}. Thus, all
elements are of the form \s{Element[type]}. Distinctions are made between four categories of elements:

\begin{enumerate}
\item \it{atomic} self contained, does not depend on another element (\s{Normal})
\item \it{compound} build out of other elements (\s{If (...), Apply (...)})
\item \it{discrete} element whose value type is discrete (\s{Poisson})
\item \it{continuous} element whose value type is continuous (\s{Gamma})
\end{enumerate}

\subsection{Observing Variables}
Elements are defined by distributions of their possible values. A
simple \s{Normal} element can only hold values on the interval
$[-1, 1]$, with corresponding probabilities. Thus, each element may be
considered a random variable, on its given distribution. Figaro
provides methodology for symbolic manipulation of these random
variables, allowing dependent variables (compound elements) to be
utilized. \\

No element is given an immediate value until it is observed. The
\s{observe()} method alters only the element on which it is called, so
that when an inference algorithm (such as
\s{VariableElimination.probability}) is run on that element, or any
variable in the same dependency network, it can calculate
the relevant conditional probabilities.

\subsection{Creating Compound Elements}
Atomic elements can be combined to form compound elements. The
simplest method for combining them is the \s{If} construct, which
allows for slightly complex conditioning on distributions

\subsubsection{If}
The type signature for \s{If} is as follows:

\begin{center}
  \s{If (test: Element[Boolean], \\ then: Element[T], \\ else:
    Element[T])\\ => cond: Element[T]}
\end{center}

\subsubsection{Apply}
The type signature for \s{Apply} is as follows:
\begin{center}
  \s{Apply (e1: Element[T], \\ fn: T -> U)}
\end{center}

\subsubsection{Chain}
The type signature for \s{Chain} is as follows:
\begin{center}
  \s{Chain ()}
\end{center}

\end{document}