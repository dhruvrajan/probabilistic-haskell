% Created 2017-07-28 Fri 08:00
\documentclass[11pt]{article}
\usepackage[utf8]{inputenc}
\usepackage[T1]{fontenc}
\usepackage{fixltx2e}
\usepackage{graphicx}
\usepackage{longtable}
\usepackage{float}
\usepackage{wrapfig}
\usepackage{rotating}
\usepackage[normalem]{ulem}
\usepackage{amsmath}
\usepackage{textcomp}
\usepackage{marvosym}
\usepackage{wasysym}
\usepackage{amssymb}
\usepackage{hyperref}
\tolerance=1000
\author{Dhruv Rajan}
\date{\today}
\title{Notes on Iterations of Distribution Representation (July 28th,  2017)}
\hypersetup{
  pdfkeywords={},
  pdfsubject={},
  pdfcreator={Emacs 25.1.1 (Org mode 8.2.10)}}
\begin{document}

\maketitle
\tableofcontents


\section{First Iteration}
\label{sec-1}
\begin{itemize}
\item Code at \url{https://github.com/dhruvrajan/probabilistic-haskell/blob/8eb317d225976c7883ff98b6ac625c87a0074036/probability/src/Distribution.hs}
\item Used \texttt{MultiParameterTypeClasses}, \texttt{FlexibleInstances}, and
\texttt{ExistentialQuantification} GHC extensions
\item \textbf{Generic Probabiltiy Distribution}
\begin{itemize}
\item Single type parameter (distribution type)
\item \texttt{sumProbabilities} method to sum the probabilities over its domain.
\end{itemize}
\item \textbf{Discrete Distribution}
\begin{itemize}
\item Typeclass with two type parameters: a distribution type, and a domain type
\begin{itemize}
\item Example: \texttt{instance Discrete Bernoulli Bool} (A Bernoulli
distribution defined over \texttt{[True, False]})
\item \texttt{pmf} (Probability Mass Function) defined over the discrete
domain
\end{itemize}
\end{itemize}
\item \textbf{Continuous Distribution}
\begin{itemize}
\item Typeclass with two type parameters: a distribution type, and a domain type
\begin{itemize}
\item Example: \texttt{instance Continuous Normal Double} vs. \texttt{instance
	Continuous Normal Float}
\end{itemize}
\item \texttt{pdf} (Probability Density Function) defined over the continuous
domain
\item \texttt{cdf} (Cumulative Distribution Function) defined over the
continuous domain
\end{itemize}
\item \textbf{Distribution Types}
\begin{itemize}
\item Each distribution is represented as its own datatype. The constructor
takes the distributions parameters in the constructor
\begin{itemize}
\item Example: (\texttt{data Normal = Normal Double Double})
\end{itemize}
\end{itemize}
\item \textbf{Issue}
\begin{itemize}
\item In this representation, the domain of a probability distribution
is kept separate from its dataype. For example, A Bernoulli
distribution is only paired with its specific domain (could be
over Bool, over Int, over String, etc.) in a typeclass instance.
\item This produces interesting, but incorrect behavior. A Bernoulli
distribution is not sufficiently defined by just its parameter,
without a domain-and thus should not be allowable. The next
iteration attemts to associate distributions with their domains
via a type parameter to their respective datatypes.
\end{itemize}
\end{itemize}
\section{Second Iteration}
\label{sec-2}
\begin{itemize}
\item Code at \url{https://github.com/dhruvrajan/probabilistic-haskell/blob/ac8197ded0bada83be2ecbc54ff509bd4847bbc4/probability/src/Distribution.hs}
\item Now only using the \texttt{Existential Quantification} GHC Extension
\item \textbf{Generic Probability Distribution}
\begin{itemize}
\item Does not have a typeclass. Probability distributions must either
be Discrete, or Continuous. The original generic typeclass was
not really of any use, so this is a cleaner distinction.
\end{itemize}
\item \textbf{Discrete Distribution}
\begin{itemize}
\item Typeclass with a single type parameter (distribution type)
\item \texttt{pmf} (Probability Mass Function) defined over some discrete domain
\begin{itemize}
\item Note the function's type signature:
\begin{itemize}
\item \texttt{pmf \textbackslash{}:: Eq b => a b -> b -> Double}
\item It requires that \texttt{a} has some type parameter \texttt{b}. Thus,
every discrete distribution datatype must specify its domain
type as a type parameter.
\item Example: \texttt{data Bernoulli a = Bernoulli Double a a} where the
constructor's first parameter is the single parameter to the
Bernoulli distribution, and the second two are the outcomes
\emph{success} and \emph{fail}
\end{itemize}
\end{itemize}
\end{itemize}
\item \textbf{Continuous Distribution}
\begin{itemize}
\item Typeclass with a single type parameter (distribution type)
\item \texttt{pmf} and \texttt{cdf} functions defined over some continuous domain
\item Note the functions' type signatures:
\begin{itemize}
\item \texttt{pdf \textbackslash{}:: (RealFloat b, Fractional b) => a b -> b -> Double}
\item \texttt{cdf \textbackslash{}:: (RealFloat b, Fractional b) => a b -> b -> Double}
\item So \texttt{b} the domain type of any continuous distribution \texttt{a}, must
be an instance of \texttt{RealFloat} and \texttt{Fractional}
\end{itemize}
\item Example: \texttt{data Normal a = Normal a a}
\begin{itemize}
\item For a normal distribution, the type parameter specifies the types
of the Normal parameters $\mu$ and $\sigma$; these could be \texttt{Double}, 
or \texttt{Float}, etc.
\end{itemize}
\end{itemize}
\item \textbf{Problems Solved}
\begin{itemize}
\item This second iteration allows for a clean representation of
multiple types of distributions. Each domain is associated with
its domain in its datatype. This will help in cases like
implementing sampling methods for each distribution. Additionally,
it makes it easy to tell whether an \texttt{Element} is discrete or continuous,
depending on its distribution.
\end{itemize}
\end{itemize}
% Emacs 25.1.1 (Org mode 8.2.10)
\end{document}