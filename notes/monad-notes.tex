\documentclass[12pt]{article}



\begin{document}

\title{Monads}
\author{Dhruv Rajan}
\date{\today}
\maketitle

\section{Background}

Functors are things that may be mapped over, in order to perform
functions on values within certain contexts. In general, operate on a
datatype \texttt{f a} with a function \texttt{a -> b} to produce a
datatype \texttt{f b}. This abstraction is encoded in the \texttt{fmap}
function: \texttt{(Functor f) => (a -> b) -> f a -> f b} \\

Applicatives are an improvement upon Functors, which allow the mapping
function to itself be wrapped inside a context. Thus, the function
\texttt{<*>} operates in the same way as \texttt{fmap}, besides this
detail, and has type \texttt{(Applicative f) => f (a -> b) -> f a -> f b}. \\

Monads can be considered an extension of Applicative Functors. While
applicatives are concerned with applying wrapped functions to wrapped
values, Monads are concerned with applying a function which /takes/ an
unwrapped value and returns a wrapped value to a wrapped value. That
is, the ability to apply a function \texttt{>>= :: (Monad m) => m a ->
  (a -> m b) -> m b} \\

\section{The Writer Monad}


\end{document}
