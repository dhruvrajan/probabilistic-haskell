\documentclass[12pt]{article}
\usepackage{amsmath}
\begin{document}

\title{Probabilistic Reasoning -- Chapter 14.}
\author{Dhruv Rajan}
\date{\today}
\maketitle

\textit{Independence} and \textit{conditional independence}
relationships are very powerful for modeling our
world. \textbf{Bayesian Networks} introduce a systematic way of
representing these relationships.

\section{Representing Knowledge in an Uncertain Domain}
Full joint probability distributions can answer any question about the
domain, however, they become intractable as the number of variables
grows. \textbf{Bayesian Networks} represent the same dependencies
among variables, and give a concise specification of \textit{any} full
joint probability distribution.

A Bayesian Network is a DAG where each node is anntated with
probability information, under the following specification:
\begin{enumerate}
\item A set of random variables makes up the nodes of the
  network. Variables may be discrete or continuous
\item A set of directed links or arrows connects pairs of nodes. If
  there is an arrow from node $X$ to node $Y$, $X$ is said to be a
  parent of Y.
\item Each Node $X_i$ has a conditional probability distribution
  $P(X_i | Parents(X_i))$ which quantifies the effect of the parents
  on the node
\item The graph has no cycles (it is a DAG)
\end{enumerate}

\section{The Semantics of Bayesian Networks}
\subsection{Bayes' nets as Representations of a Full Joint Distribution}
A generic query to a full joint distribution (FJD) is the probability
of a conjunction of assignments to each variable:
$P(X_1 = x_1 \land \dots \land X_n = x_n)$. The following formula can
be used.
\begin{align*}
  P(x_1, \dots x_n) = \prod_{i = 1}^{n} P(x_i | parents(X_i))
\end{align*}
So the Bayesian network can answer any query, simply by summing the
relevant joint CPDs for each node.
\end{document}
