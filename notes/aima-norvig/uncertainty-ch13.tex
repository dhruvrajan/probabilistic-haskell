\documentclass[12pt]{article}
\begin{document}

\title{Chapter 13---Uncertainty}
\author{Dhruv Rajan}
\date{\today}
\maketitle

\section{Acting Under Uncertainty}

Logical agents act under \textit{undcertainty} for they do not possess
the full truth about their environments. Agents act to maximize
\textit{performance measures}, and must construct/choose rules
accordingly. A \textit{rational decision}, thus, depends on the
relative importance of various goals, and the likelihood and degree to
which they will be achieved.

\subsection{Encoding Uncertain Knowledge}

Knowledge is often expressed using rules and propositions from
\textit{first order logic}. Thus, we can express generalizations and
instantiations as universal truths. These are, effectively, causal
relationships. When knowledge is uncertain, it is impossible to
exhaustively enumerate the set of rules, since infinite causes can be
attributed to any effect, and vice-versa (this is the problem of
\textbf{laziness}). There is also the problem of
\textbf{ignorance}---we may not know all the rules anyway, or have any
way of being certain about them. Probability provides a way of
summarizing the uncertainty which comes from our laziness and
ignorance, allowing us to express a \textbf{degree of belief} in the
accuracy / completeness of our rule set. It also allows us to update
our rule set (set of beliefs about the world) by observing evidence, transforming \textit{prior} notions to \textit{posterior} notions.

\section{Probability Notation}

The basic element of probabilistic statements is the \textbf{random
  variable}. It refers to a ``part'' of the world whose ``status'' is
initially unknown. A random variable is restricted to a domain of
possible values: There are \textbf{Boolean, Discrete, and Continuous}
random variables.

\subsection{Atomic Events}
An \textit{atomic event} is a complete specification of the world of
which the agent is uncertain. The world is modeled by a set of random
variables, and thus the \textit{atomic event} is an assignment of
values to this entire set.
\begin{itemize}
\item atomic events must be mutually exclusive
\item The set of all possible atomic events must be exhaustive
\item Any particular atomic event entails the truth value of
  \textit{every} proposition
\item A proposition is a disjunction of all atomic events which entail
  the proposition
\end{itemize}

\subsection{Prior Probability}
The \textit{prior probability} of a proposition is the degree of
belief that it will be true, in the absence of any other information.
\end{document}